\pagebreak

\section*{Sažetak}


Ovo je \LaTeX\ predložak za izradu završnog rada prvog ciklusa studija, izrađen za potrebe studenata Elektrotehničkog fakulteta u Sarajevu. U radu se isprepliću dvije odvojene cjeline - sadržaj i forma. Sadržaj rada je baziran na idejama iz važećeg Pravilnika o strukturi i sadržaju doktorske disertacije i magistarskog rada na Elektrotehničkom fakultetu u Sarajevu (br. 04-1-673/11, dana 17.01.2011. godine), dok je forma rada definirana strukturom i načinom korištenja .tex fajlova.
U fajlu Zavrsni\_rad\_BSc\_Ime\_Prezime.tex oblikovane su osnovne stranice, a naredbom \textit{include} umeću se dodatne stranice i dodaju poglavlja. 

Osim \textit{Zavrsni\_rad\_BSc\_Ime\_Prezime.tex} fajla, koristi se još i: \textit{abstract.tex} (izdvojen fajl za sažetak na bosanskom i engleskom jeziku), \textit{postavka.tex} (izdvojen fajl za postavku rada), \textit{izjava.tex} (izdvojen fajl za izjavu o autentičnosti rada), \textit{poglavlje\_1.tex} (primjer jednog poglavlja),\textit{ prilog\_1.tex} (primjer jednog priloga) i \textit{literatura.bib} (bibliografski podaci).

U sažetku je potrebno dati koncizan opis riješenih problema, metoda korištenih za njegovo (njihovo) rješavanje, dobivenih rezultata i zaključke. U sažetku se ne navode reference. Potrebno je voditi računa da se u sažetku ne daje uvod u rad, niti pregled poglavlja rada, već daje opis namjene rada do najviše 500 riječi.

\vspace{1cm}
\textbf{Ključne riječi}:  predložak, \LaTeX, ETF

\section*{Abstract}

The section "Sažetak" should actually be translated to the section "Abstract". Please avoid the direct usage of google-translate.  

\vspace{1cm}
\textbf{Keywords}:  template, \LaTeX , ETF