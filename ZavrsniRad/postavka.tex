%%%%%%%%%%%%%%%%%%%% POSTAVKA RADA %%%%%%%%%%%%%%%%%%%%%%%%%%%%%%%
\thispagestyle{plain}
\begin{flushleft}
\textbf{Elektrotehnički fakultet, Univerzitet u Sarajevu}\\
\textbf{Odsjek za ..............}\\
\textbf{Vanr. prof. dr Alessandro Volta, dipl.el.ing}\\


\textbf{Sarajevo, .........(datum) }\\
\end{flushleft}

\begin{center}
\vspace{2cm}
{\Large Postavka zadatka završnog rada I ciklusa:}\\
\vspace{0.2cm}
{\large \textbf{Predložak za izradu završnog rada I ciklusa - uz korištenje Latexa kao alata}}
\end{center}

\vspace{0.5cm}
U okviru rada je potrebno razviti \LaTeX\ predložak za izradu završnog rada prvog ciklusa studija. U radu je potrebno:
\begin{itemize}
\item objasniti opći postupak za izradu i pisanje završnih radova prvog ciklusa studija, poštivajući odgovarajuće Pravilnike,
\item dati kratko uputstvo za korištenje .tex dokumenata, pisanje slika, tabela i relacija, generisanje sadržaja, popisa slika i tabela i sl.
\item dati kratko upusto za generiranje odgovarajućih .pdf dokumenata iz odgovarajućih .tex fajlova,
\end{itemize}
Preporučuje se korištenje TexLive \LaTeX\ podrške, verzije 2013 ili novije.

\textbf{Napomena: U dijelu "Postavka" se stavlja postavka rada koju je specificirao mentor prilikom davanja teme. Naročito obratiti pažnju na naslov rada koji mora biti konzistentan tokom cijelog dokumenta, i usaglašen sa temom upisanom u informacioni sistem (npr. Zamger). Na ovoj stranici se potpisuje mentor prije nego se radovi odnesu u studentsku službu.}


\vspace{1cm}
\textbf{Polazna literatura:}

\begin{itemize}
\item[]  [1] Sonka, M., Hlavec, V., Boyle, R., "Image Processing, Analysis, and Machine Vision", Thomson, 2008.
\item[] [2] Lee, E.A., Varaiya, P., "Structure and Interpretation of Signals and Systems", Electrical Engineering \& Computer Science University of California, Berkeley, 2000

\end{itemize}


\begin{center}
\vspace{2cm}
\makebox[8cm]{\hrulefill} \\
Vanr. prof. dr Alessandro Volta, dipl. ing. el.\\
\end{center}





